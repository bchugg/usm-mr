\documentclass{beamer}
\usetheme{Hannover}
\usefonttheme{serif}
\usepackage{graphicx}
\usepackage{algpseudocode}

\newcommand{\R}{\mathbb{R}}
\newcommand{\am}{\text{argmax}}

\title{Unconstrained Submodular Maximization in MapReduce}
\author{Sikander Randhawa, Ben Chugg, Angad Kalra}
\institute{\textsc{The university of British Columbia \\ Vancouver, Canada}}
\date{}


\begin{document}

\begin{frame}
\titlepage
\end{frame}

\begin{frame}
\section{Background}
\frametitle{Background}
Formally, a set function $f:2^X\rightarrow \R$ is \textit{submodular} iff 
\[f(A\cup \{x\})-f(A) \geq f(B\cup\{x\})-f(B),\]
for all $A\subset B$. \\
\vspace{0.4cm}
Intuitively, think of decreasing marginal gains: The more you have, the less you appreciate gaining something new. \\
\vspace{0.4cm}
These functions find applications in graph theory, game theory and machine learning. 
\end{frame}

\begin{frame}
\frametitle{Background}
We are interested in maximizing such functions, i.e., 
\[\max_{S\subset X}f(S).\]
There are no constraints on the subsets over which we can maximize and the functions are non-monotone --- \textit{unconstrained, non-monotone,  submodular maximization (USM)}. \\
\vspace{0.4cm}
In general, this problem is NP-Hard so we search for approximations. 
\end{frame}

\begin{frame}
\section{Local Search}
\frametitle{Local Search}
Local Search (LS) achieves a $(1/3-\frac{\epsilon}{n^2})$-approximation in a centralized setting. The idea: Search for elements that increase the marginal by a factor of $(1+\frac{\epsilon}{n})$, either by adding them to, or removing them from, the current solution. \\
\vspace{0.3cm}
\textbf{Local Search}
\begin{enumerate}
\item[1] $v\gets \max_{v\in X}f(\{v\})$, $S\gets\{v\}$ 
\item[2] If there exists an element $x\in X\setminus S$ such that $f(S\cup \{x\})\geq (1+\frac{\epsilon}{n})f(S)$ then $S\gets S\cup\{x\}$. Go back to step 2. 
\item[3] If there exists an element $s\in S$ such that $f(S\setminus\{s\})\geq (1+\frac{\epsilon}{n})f(S)$ then $S\gets S\setminus \{x\}$. Go back to step 2. 
\item[4] Return $\am\{f(S),f(X\setminus S)\}$. 
\end{enumerate}
\end{frame}

\begin{frame}
We want to parallelize Local Search. \\

I don't really know how to write this without having way too much detail. 
\end{frame}

\begin{frame}
\section{Results}
Experiments were implemented in \textsc{Matlab}. \\
\vspace{0.4cm}
Parallelization was mimicked --- the algorithm was not truly implemented in parallel. \\
\vspace{0,4cm}
It was also not implemented in MapReduce. 
\end{frame}

\begin{frame}
\subsection{Number of Rounds}

\frametitle{Number of Rounds}
\includegraphics[scale=0.5]{../plots/{CUTFUNC_ROUNDS_N_100:50:600_eps_0.1_p_log(n)}.png}
\end{frame}

\begin{frame}
\subsection{Function Value}
\frametitle{Function Value}
\includegraphics[scale=0.5]{../plots/{CUTFUNC_FUNCVALS_rounds_20_n_600_eps_0.1_p_log(n)}.png}
\end{frame}

\begin{frame}
\subsection{Improvements}
\frametitle{Number of Improvements}
\includegraphics[scale=0.5]{../plots/{CUTFUNC_IMPRVMTS_rounds_20_n_600_eps_0.1_p_log(n)}.png}
\end{frame}

\begin{frame}
\section{Results}
\frametitle{What the f*** is going on?}
Woooo! 
\end{frame}

\end{document}
